\section{Pseudorandom-number generators}

\index{random@\q{random}}
\index{random-number generator}
\index{pseudorandom-number generator}
We see that unlike \q{add2}, the procedure \q{bump-counter}
returns a different result each time it’s called, because it
side-effects something outside itself. One particularly useful
variant of this procedure generates a different {\em random}
number each time it’s called. Many Schemes provide this as a
primitive procedure called \q{random}\f{Writing your own version
of \q{random} in Scheme requires quite a bit of mathematical
chops to get something acceptable. We won’t get into that here.}:
When called with no argument, \q{(random)} returns a
“pseudorandom” number in the interval [0, 1), i.e., between 0
(inclusive) and 1 (exclusive), such that the results are
uniformly distributed within that interval.

\q{
(random) |evalsto 0.6360226737551197
(random) |evalsto 0.8057127493871963
(random) |evalsto 0.8595213305159558
}

\n With only a little bit of arithmetic, \q{random} can be used
to simulate events of known probability. Consider the throwing of
a fair, six-headed die: The outcomes 1, 2, 3, 4, 5, 6 are equally
likely. To simulate this, we divide the interval [0, 1) into six
equal segments, associate each segment with an outcome, and have
\q{(random)}’s output decide which outcome occurred. One way to
do it is to multiply the random number $n$ ($0 \le n < 1$) by 6
and take the ceiling: I'll leave you to convince yourself that
this takes on the value of each die face with 1/6 probability.

\scmfilename dice.scm

\scmdribble{
(define throw-one-die
  (lambda ()
    (let ((result (ceiling (* (random) 6))))
      result)))
}

\n \q{throw-two-dice} simply calls \q{throw-one-die} twice:

\scmdribble{

(define throw-two-dice
  (lambda ()
    (+ (throw-one-die) (throw-one-die))))
}

\n It returns an integer between 2 and 12 inclusive — not all
equally likely!
