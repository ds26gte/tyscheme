\scmfilename obj1.scm

\scmwrite{
(load-relative "deldup.scm")
(load-relative "defstruct.scm")
}

\section{A simple object system}

\index{inheritance!single}
Let us now implement a basic object system in Scheme.
We will allow only one superclass per class ({\em
single inheritance}).  If we don’t want to specify a
superclass, we will use \q{#t} as a “zero”
superclass, one that has neither slots nor methods.
The superclass of \q{#t} is deemed to be itself.

As a first approximation, it is useful to define
classes using a struct called \q{standard-class}, with
fields for the slot names, the superclass, and the
methods.  The first two fields we will call \q{slots}
and \q{superclass} respectively.  We will use {\em two}
fields for methods, a
\q{method-names} field that will hold the list of names
of the class’s methods, and a
\q{method-vector} field that will hold the vector of
the values of the class’s methods.\f{We could in theory
define methods also as slots (whose values happen to be
procedures), but there is a good reason not to.  The
instances of a class share methods but in general
differ in their slot values.  In other words, methods
can be included in the class definition and don’t have
to be allocated per instance as slots have to be.}
Here is the definition of the \q{standard-class}:

\scmdribble{
(defstruct standard-class
  slots superclass method-names method-vector)
}

\n We can use \q{make-standard-class}, the maker procedure of
\q{standard-class}, to create a new class.  E.g.,

\q{
(define trivial-bike-class
  (make-standard-class
   'superclass #t
   'slots '(frame parts size)
   'method-names '()
   'method-vector #()))
}

\n This is a very simple class.   More complex classes
will have non-trivial superclasses and methods, which
will require a lot of standard initialization that we
would like to hide within the class creation process.
We will therefore define a macro called
\q{create-class} that will make the appropriate call to
\q{make-standard-class}.

\scmdribble{
(define-macro create-class
  (lambda (superclass slots . methods)
    `(create-class-proc
      ,superclass
      (list ,@(map (lambda (slot) `',slot) slots))
      (list ,@(map (lambda (method) `',(car method)) methods))
      (vector ,@(map (lambda (method) `,(cadr method)) methods)))))
}

\n We will defer the definition of the
\q{create-class-proc} procedure to later. 

The procedure \q{make-instance} creates an {\em
instance} of a class by generating a fresh vector based
on information enshrined in the class.  The format of
the instance vector is very simple: Its first element
will refer to the class, and its remaining elements
will be slot values.  \q{make-instance}’s arguments are
the class followed by a sequence of twosomes, where
each twosome is a slot name and the value it assumes in
the instance.

\scmdribble{
(define make-instance
  (lambda (class . slot-value-twosomes)

    ;Find ‘n’, the number of slots in ‘class’.
    ;Create an instance vector of length ‘n + 1’,
    ;because we need one extra element in the instance
    ;to contain the class.

    (let* ((slotlist (standard-class.slots class))
           (n (length slotlist))
           (instance (make-vector (+ n 1))))
      (vector-set! instance 0 class)

      ;Fill each of the slots in the instance
      ;with the value as specified in the call to
      ;‘make-instance’.

      (let loop ((slot-value-twosomes slot-value-twosomes))
        (if (null? slot-value-twosomes) instance
            (let ((k (list-position (car slot-value-twosomes) 
                                    slotlist)))
              (vector-set! instance (+ k 1) 
                (cadr slot-value-twosomes))
              (loop (cddr slot-value-twosomes))))))))
}

\n Here is an example of instantiating a class:

\q{
(define my-bike
  (make-instance trivial-bike-class
                 'frame 'cromoly
                 'size '18.5
                 'parts 'alivio))
}

\n This  binds \q{my-bike} to the instance

\q{
#(<trivial-bike-class> cromoly alivio 18.5)
}

\n where \q{<trivial-bike-class>} is a Scheme datum (another
vector) that is the value of \q{trivial-bike-class}, as defined
above.

The procedure \q{class-of} returns the class of an instance:

\q{
(define class-of
  (lambda (instance)
    (vector-ref instance 0)))
}

\n This assumes that \q{class-of}’s argument will be a class
instance, i.e., a vector whose first element points to some
instantiation of the \q{standard-class}.
We probably want to make \q{class-of} return an appropriate value
for any kind of Scheme object we feed to it.

\scmdribble{
(define class-of
  (lambda (x)
    (if (vector? x)
        (let ((n (vector-length x)))
          (if (>= n 1)
              (let ((c (vector-ref x 0)))
                (if (standard-class? c) c #t))
              #t))
        #t)))
}

\n The class of a Scheme object that isn’t created
using \q{standard-class} is deemed to be \q{#t}, the
zero class.

The procedures \q{slot-value} and \q{set!slot-value}
access and mutate the values of a class instance:

\scmdribble{
(define slot-value
  (lambda (instance slot)
    (let* ((class (class-of instance))
           (slot-index
            (list-position slot (standard-class.slots class))))
      (vector-ref instance (+ slot-index 1)))))

(define set!slot-value
  (lambda (instance slot new-val)
    (let* ((class (class-of instance))
           (slot-index
            (list-position slot (standard-class.slots class))))
      (vector-set! instance (+ slot-index 1) new-val))))
}

\n We are now ready to tackle the definition of
\q{create-class-proc}.  This procedure takes a
superclass, a list of slots, a list of method names,
and a vector of methods and makes the appropriate call
to \q{make-standard-class}.  The only tricky part is
the value to be given to the \q{slots} field.  It can’t
be just the slots argument supplied via
\q{create-class}, for a class must include the slots of
its superclass as well.  We must append the supplied
slots to the superclass’s slots, making sure that we
don’t have duplicate slots.

\scmdribble{
(define create-class-proc
  (lambda (superclass slots method-names method-vector)
    (make-standard-class
     'superclass superclass
     'slots
     (let ((superclass-slots 
            (if (not (eqv? superclass #t))
                (standard-class.slots superclass)
                '())))
       (if (null? superclass-slots) slots
           (delete-duplicates
            (append slots superclass-slots))))
     'method-names method-names
     'method-vector method-vector)))
}

\index{delete-duplicates@\q{delete-duplicates}}
\n The procedure \q{delete-duplicates} called on a list
\q{s}, returns a new list that only includes the {\em last}
occurrence of each element of \q{s}.

\q{
(define delete-duplicates
  (lambda (s)
    (if (null? s) s
        (let ((a (car s)) (d (cdr s)))
          (if (memv a d) (delete-duplicates d)
              (cons a (delete-duplicates d)))))))
}

Now to the application of methods.  We invoke the
method on an instance by using the procedure \q{send}.
\q{send}’s arguments are the method name, followed by
the instance, followed by any arguments the method has
in addition to the instance itself.  Since methods are
stored in the instance’s class instead of the instance
itself, \q{send} will search the instance’s class for
the method.  If the method is not found there, it is
looked for in the class’s superclass, and so on further
up the superclass chain:

\scmdribble{
(define send
  (lambda (method instance . args)
    (let ((proc
           (let loop ((class (class-of instance)))
             (if (eqv? class #t) (error 'send)
                 (let ((k (list-position 
                           method
                           (standard-class.method-names class))))
                   (if k
                       (vector-ref (standard-class.method-vector class) k)
                       (loop (standard-class.superclass class))))))))
      (apply proc instance args))))
}

We can now define some more interesting classes:

\scmfilename bike.scm

\scmdribble{
(define bike-class
  (create-class
   #t
   (frame size parts chain tires)
   (check-fit (lambda (me inseam)
                (let ((bike-size (slot-value me 'size))
                      (ideal-size (* inseam 3/5)))
                  (let ((diff (- bike-size ideal-size)))
                    (cond ((<= -1 diff 1) 'perfect-fit)
                          ((<= -2 diff 2) 'fits-well)
                          ((< diff -2) 'too-small)
                          ((> diff 2) 'too-big))))))))
}

\n Here, \q{bike-class} includes a method \q{check-fit}, that
takes a bike and an inseam measurement and reports on
the fit of the bike for a person of that inseam.

Let’s redefine \q{my-bike}:

\scmdribble{
(define my-bike
  (make-instance bike-class
                 'frame 'titanium ; I wish
                 'size 21
                 'parts 'ultegra
                 'chain 'sachs
                 'tires 'continental))
}

To check if this will fit someone with inseam 32:

\q{
(send 'check-fit my-bike 32)
}

We can subclass \q{bike-class}.

\scmdribble{
(define mtn-bike-class
  (create-class
    bike-class
    (suspension)
    (check-fit (lambda (me inseam)
                (let ((bike-size (slot-value me 'size))
                      (ideal-size (- (* inseam 3/5) 2)))
                  (let ((diff (- bike-size ideal-size)))
                    (cond ((<= -2 diff 2) 'perfect-fit)
                          ((<= -4 diff 4) 'fits-well)
                          ((< diff -4) 'too-small)
                          ((> diff 4) 'too-big))))))))
}

\n \q{mtn-bike-class} adds a slot called \q{suspension}
and uses a slightly different definition for the method
\q{check-fit}.

\scmfilename deldup.scm

\scmwrite{
(define delete-duplicates
  (lambda (s)
    (if (null? s) s
        (let ((a (car s)) (d (cdr s)))
          (if (memv a d) (delete-duplicates d)
              (cons a (delete-duplicates d)))))))
}

