\chapter{A clock for infinity}
\label{clock}

\index{clock!Guile}
\index{Guile!clock}
The Guile \cite{guile} procedure \q{alarm} provides an interruptable
timer mechanism.  The user can set or reset the alarm
for some time units, or stop it.  When the alarm’s
timer runs out of this time, it will set off an alarm,
whose consequences are user-settable.  Guile’s
\q{alarm} is not quite the clock of
sec \ref{engine-clock}, but we can modify it easily
enough.

The alarm’s timer is initially {\em stopped} or {\em
quiescent}, i.e., it will not set off an alarm even as
time goes by.  To set the alarm’s time-to-alarm to be
\q{n} seconds, where \q{n} is not \q{0}, run \q{(alarm
n)}.  If the timer was already set (but has not yet set
off an alarm), the \q{(alarm n)} procedure call will
return the number of seconds remaining from the
previous alarm setting.  If there is no previous alarm
setting, \q{(alarm n)} returns \q{0}.

The procedure call \q{(alarm 0)} {\em stops} the
alarm’s timer, i.e., the countdown of time is stopped,
the timer becomes quiescent and no alarm will go off.
\q{(alarm 0)} also returns the seconds remaining from a
previous alarm setting, if any.

By default, when the alarm’s countdown reaches 0,
Guile will display a message on the console and exit.
More useful behavior can be obtained by
using the procedure
\q{sigaction}, as follows:

\q{
(sigaction SIGALRM
  (lambda (sig)
    (display "Signal ")
    (display sig)
    (display " raised.  Continuing...")
    (newline)))
}

\n
The first argument \q{SIGALRM} (which happens to be
\q{14}) identifies to \q{sigaction} that it is the
alarm handler that needs setting.\f{There are other
signals with their corresponding handlers, and
\q{sigaction} can be used to set these as well.}  The
second argument is a unary alarm-handling procedure of
the user’s choice.  In this example, when the alarm
goes off, the handler displays \q{"Signal 14 raised.
Continuing..."} on the console without exiting Scheme.
(The \q{14} is the \q{SIGALRM} value that the alarm
will pass to its handler.  Don’t worry about it now.)

From our point of view, this simple timer mechanism
poses one problem.  A return value of \q{0} from a call
to the procedure \q{alarm} is ambiguous: It could
either mean that the alarm was quiescent, or that 
it was just about to run out of time.  We could resolve
this ambiguity if we could include “\q{*infinity*}”
in the alarm arithmetic.  In other words, we would like
a {\em clock} that works almost like \q{alarm},
except that a quiescent clock is one with
\q{*infinity*} seconds.  This will make many things
natural, viz.,

(1) \q{(clock n)} on a quiescent clock returns
    \q{*infinity*}, not \q{0}.

(2) To stop the clock, call \q{(clock *infinity*)},
     {\em not} \q{(clock 0)}.

(3) \q{(clock 0)} is equivalent to setting the clock to
an infinitesimally small amount of time, viz., to cause
it to raise an alarm instantaneously.

In Guile, we can define \q{*infinity*} as the following
“number”:

\scmfilename guile-clock.scm
\scmdribble{
(define *infinity* (/ 1 0))
}

We can define \q{clock} in terms of \q{alarm}.

\scmdribble{
(define clock
  (let ((stopped? #t)
        (clock-interrupt-handler
         (lambda () (error "Clock interrupt!"))))
    (let ((generate-clock-interrupt
           (lambda ()
             (set! stopped? #t)
             (clock-interrupt-handler))))
      (sigaction SIGALRM
                 (lambda (sig) (generate-clock-interrupt)))
      (lambda (msg val)
        (case msg
          ((set-handler)
           (set! clock-interrupt-handler val))
          ((set)
           (cond ((= val *infinity*)
                  ;This is equivalent to stopping the clock.
                  ;This is almost equivalent to (alarm 0), except
                  ;that if the clock is already stopped,
                  ;return *infinity*.

                  (let ((time-remaining (alarm 0)))
                    (if stopped? *infinity*
                        (begin (set! stopped? #t)
                          time-remaining))))

                 ((= val 0)
                  ;This is equivalent to setting the alarm to
                  ;go off immediately.  This is almost equivalent
                  ;to (alarm 0), except you force the alarm
                  ;handler to run.

                  (let ((time-remaining (alarm 0)))
                    (if stopped?
                        (begin (generate-clock-interrupt)
                          *infinity*)
                        (begin (generate-clock-interrupt)
                          time-remaining))))

                 (else
                  ;This is equivalent to (alarm n) for n != 0.
                  ;Just remember to return *infinity* if the
                  ;clock was previously quiescent.

                  (let ((time-remaining (alarm val)))
                    (if stopped?
                        (begin (set! stopped? #f) *infinity*)
                        time-remaining))))))))))
}

\n
The \q{clock} procedure uses three internal state
variables:

(1) \q{stopped?}, to describe if the clock is stopped;

(2) \q{clock-interrupt-handler}, which is a thunk
describing the user-specified part of the
alarm-handling action; and

(3) \q{generate-clock-interrupt}, another thunk which
will set \q{stopped?} to false before running the
user-specified alarm handler.

The \q{clock} procedure takes two arguments.  If the
first argument is \q{set-handler}, it uses the second
argument as the alarm handler.

If the first argument is \q{set}, it sets the
time-to-alarm to the second argument, returning the
time remaining from a previous setting.  The code
treats \q{0}, \q{*infinity*} and other values for time
differently so that the user gets a mathematically
transparent interface to \q{alarm}.

\scmwrite{
(define clock-min min)
(define clock-minus -)
(define clock-plus +)
}
