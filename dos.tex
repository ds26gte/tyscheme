\chapter{DOS batch files in Scheme}
\label{dos}

\index{script!DOS}
DOS shell scripts are known as {\em batch files}.  A
conventional DOS batch file that outputs ``Hello,
World!''\ has the following contents:

\p{
echo Hello, World!
}

It uses the DOS command \p{echo}.  The batch file is
named \p{hello.bat}, which identifies it to the
operating system as an executable.  It may then be
placed in one of the directories on the \p{PATH}
environment variable.  Thereafter, anytime one types

\p{
hello.bat
}

or simply

\p{
hello
}

\n at the DOS prompt, one promptly gets the
insufferable greeting.

A Scheme version of the hello batch file will perform
the same output using Scheme, but we need something in
the file to inform DOS that it needs to construe the
commands in the file as Scheme, and not as its default
batch language.  The Scheme batch file, also called
\p{hello.bat}, looks like:

\p+
;@echo off
;goto :start
#|
:start
echo. > c:\_temp.scm
echo (load (find-executable-path "hello.bat" >> c:\_temp.scm
echo "hello.bat")) >> c:\_temp.scm
mzscheme -r c:\_temp.scm %1 %2 %3 %4 %5 %6 %7 %8 %9
goto :eof
|#

(display "Hello, World!")
(newline)

;:eof
+

The lines upto \p+|#+ are standard DOS batch.  Then
follows the Scheme code for the greeting.  Finally,
there is one more standard DOS batch line, viz,
\p{;:eof}.  

When the user types \p{hello} at the DOS prompt, DOS
reads and runs the file \p{hello.bat} as a regular
batch file.  The first line, \p{;@echo off}, turns off
the echoing of the commands run --- as we don't want
excessive verbiage clouding the effect of our script.
The second line, \p{;goto :start}, causes execution to
jump forward to the line labeled \p{:start}, ie, the
fourth line.  The three ensuing \p{echo} lines create a
temporary Scheme file called \p{c:\_temp.tmp} with the
following contents:

\q{
(load (find-executable-path "hello.bat" "hello.bat"))
}

The next batch command is a call to MzScheme.  The
\p{-r} option loads the Scheme file \p{c:\_temp.scm}.
All the arguments (in this example, none) will be
available to Scheme in the vector \q{argv}.  This call
to Scheme will evaluate our Scheme script, as we will
see below.  After Scheme returns, we still need to
ensure that the batch file winds up cleanly.  The next
batch command is \p{goto :eof}, which causes
control to skirt all the Scheme code and go to the very
end of the file, which contains the label
\p{;:eof}.   The script thus ends.

Now we can see how the call to Scheme does its part,
viz, to run the Scheme expressions embedded in the
batch file.  Loading \p{c:\_temp.scm} will cause Scheme
to deduce the full pathname of the file \p{hello.bat}
(using \q{find-executable-path}), and to then {\em
load}
\p{hello.bat}.  

Thus, the Scheme script file will now be run as a
Scheme file, and the Scheme forms in the file will have
access to the script's original arguments via the
vector \q{argv}.

Now, Scheme has to skirt the batch commands in the
script.  This is easily done because these batch
commands are either prefixed with a semicolon or are
enclosed in \p+#| ... |#+, making them Scheme comments.

The rest of the file is of course straight Scheme, and
the expressions therein are evaluated in sequence.  (The
final expression, \p{;:eof}, is a Scheme comment, and
causes no harm.)  After all the expressions have been
evaluated, Scheme will exit.

In sum, typing \p{hello} at the DOS prompt will produce

\p{
Hello, World!
}

\n and return you to the DOS prompt.
