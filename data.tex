\chapter{Data types}

\index{data type}
A {\em data type} is a collection of related values.
These collections
need not be disjoint, and they are often hierarchical.
Scheme has a rich set of data types: some are simple
(indivisible) data types and others are compound data types
made by combining other data types.

\index{data type!simple}

\section{Simple data types}

The simple data types of Scheme include booleans, numbers,
characters, and symbols.


\index{boolean}

\subsection{Booleans}
\label{booleans}

\index{boolean?@\q{boolean?}}
\index{t@\q{#t}}
\index{f@\q{#f}}
\index{truth}
\index{falsity}
Scheme’s booleans are \q{#t} for true and \q{#f} for false.
Scheme has a predicate procedure called \q{boolean?} that
checks if its argument is boolean.

\q{
(boolean? #t)              |evalsto #t
(boolean? "Hello, World!") |evalsto #f
}

\index{not@\q{not}}

\n The procedure \q{not} negates its argument, considered as a
boolean.

\q{
(not #f)              |evalsto #t
(not #t)              |evalsto #f
(not "Hello, World!") |evalsto #f
}

\n The last expression illustrates a Scheme convenience:
In a context that requires a boolean, Scheme will treat
any value that is not \q{#f} as a true value.


\index{number}

\subsection{Numbers}

Scheme numbers can be integers (e.g., \q{42}), rationals
(\q{22/7}), reals (\q{3.1416}), or complex (\q{2+3i}).  An
integer is a rational is a real is a complex number is a
number.  Predicates exist for testing the various kinds of
numberness:

\index{number?@\q{number?}}
\index{integer?@\q{integer?}}
\index{rational?@\q{rational?}}
\index{real?@\q{real?}}
\index{complex?@\q{complex?}}

\q{
(number? 42)       |evalsto #t
(number? #t)       |evalsto #f
(complex? 2+3i)    |evalsto #t
(real? 2+3i)       |evalsto #f
(real? 3.1416)     |evalsto #t
(real? 22/7)       |evalsto #t
(real? 42)         |evalsto #t
(rational? 2+3i)   |evalsto #f
(rational? 3.1416) |evalsto #t
(rational? 22/7)   |evalsto #t
(integer? 22/7)    |evalsto #f
(integer? 42)      |evalsto #t
}

\index{b@\q{#b} (binary number)}
\index{o@\q{#o} (octal number)}
\index{x@\q{#x} (hexadecimal number)}
\index{d@\q{#d} (decimal number)}

\n Scheme integers need not be specified in decimal (base 10)
format.  They can be specified in binary by prefixing the
numeral with \q{#b}.  Thus \q{#b1100} is the number twelve.
The octal prefix is \q{#o} and the hex prefix is
\q{#x}.  (The optional decimal prefix is \q{#d}.)

\index{eqv?@\q{eqv?}}

Numbers can tested for equality using the general-purpose
equality predicate \q{eqv?}.

\q{
(eqv? 42 42)   |evalsto #t
(eqv? 42 #f)   |evalsto #f
(eqv? 42 42.0) |evalsto #f
}

\index{=@\q{=}}

\n However, if you know that the arguments to be compared are
numbers, the special number-equality predicate \q{=} is more
apt.

\q{
(= 42 42)   |evalsto #t
(= 42 #f)   |causeserror
(= 42 42.0) |evalsto #t
}

\index{<@\q{<}}
\index{<=@\q{<=}}
\index{>@\q{>}}
\index{>=@\q{>=}}

\n Other number comparisons allowed are
\q{<}, \q{<=}, \q{>}, \q{>=}.

\q{
(< 3 2)    |evalsto #f
(>= 4.5 3) |evalsto #t
}

\index{+@\q{+}}
\index{-@\q{-}}
\index{*@\q{*}}
\index{/@\q{/}}
\index{expt@\q{expt}}

\n Arithmetic procedures \q{+}, \q{-}, \q{*}, \q{/}, \q{expt} have the
expected behavior:

\q{
(+ 1 2 3)    |evalsto 6
(- 5.3 2)    |evalsto 3.3
(- 5 2 1)    |evalsto 2
(* 1 2 3)    |evalsto 6
(/ 6 3)      |evalsto 2
(/ 22 7)     |evalsto 22/7
(/ 22.0 7.0)     |evalsto 3.14285...
(expt 2 3)   |evalsto 8
(expt 4 1/2) |evalsto 2.0
}

\n For a single argument, \q{-} and \q{/} return the negation
and the reciprocal respectively:

\q{
(- 4) |evalsto -4
(/ 4) |evalsto 1/4
}

\index{max@\q{max}}
\index{min@\q{min}}

\n The procedures \q{max} and \q{min} return the maximum and
minimum respectively of the number arguments supplied to
them.  Any number of arguments can be so supplied.

\q{
(max 1 3 4 2 3) |evalsto 4
(min 1 3 4 2 3) |evalsto 1
}

\index{abs@\q{abs}}

\n The procedure \q{abs} returns the absolute value of
its argument.

\q{
(abs  3) |evalsto 3
(abs -4) |evalsto 4
}

\index{round@\q{round}}
\index{ceiling@\q{ceiling}}
\index{floor@\q{floor}}

\n The procedure \q{round} returns the closest integer to the
argument. The procedures \q{ceiling} and \q{floor} return the nearest
integer above and below respectively.

\q{
(round 2.718)   |evalsto 3
(ceiling 2.718) |evalsto 3
(floor 2.718)   |evalsto 2
}

\index{atan@\q{atan}}
\index{exp@\q{exp}}
\index{sqrt@\q{sqrt}}

\n This is just the tip of the iceberg.  Scheme
provides a large and comprehensive suite of arithmetic
and trigonometric procedures.  For instance, \q{atan},
\q{exp}, and \q{sqrt} respectively return the
arctangent, natural antilogarithm, and 
square root of their argument.  Consult
R5RS \cite{r5rs} for more details.

\index{character}

\subsection{Characters}

\index{character!-notation@\p+#\+ notation for}
Scheme character data are represented by prefixing the
character with \p{#\}.  Thus, \q{#\c} is the character
\p{c}.  Some non-graphic characters have more descriptive
names, e.g., \q{#\newline}, \q{#\tab}.  The character for
space can be written \q{#\ }~,  or more readably, \q{#\space}.

\index{char?@\q{char?}}

The character predicate is \q{char?}:

\q{
(char? #\c) |evalsto #t
(char? 1)   |evalsto #f
(char? #\;) |evalsto #t
}

\n Note that a semicolon character datum does not trigger
a comment.

\index{char=?@\q{char=?}}
\index{char<?@\q{char<?}}
\index{char<=?@\q{char<=?}}
\index{char>?@\q{char>?}}
\index{char>=?@\q{char>=?}}

The character data type has its set of comparison
predicates: \q{char=?}, \q{char<?}, \q{char<=?}, \q{char>?},
\q{char>=?}.

\q{
(char=? #\a #\a)  |evalsto #t
(char<? #\a #\b)  |evalsto #t
(char>=? #\a #\b) |evalsto #f
}

\index{char-ci=?@\q{char-ci=?}}
\index{char-ci<?@\q{char-ci<?}}
\index{char-ci<=?@\q{char-ci<=?}}
\index{char-ci>?@\q{char-ci>?}}
\index{char-ci>=?@\q{char-ci>=?}}

\n To make the comparisons case-insensitive, use \q{char-ci}
instead of \q{char} in the procedure name:

\q{
(char-ci=? #\a #\A) |evalsto #t
(char-ci<? #\a #\B) |evalsto #t
}

\index{char-downcase@\q{char-downcase}}
\index{char-upcase@\q{char-upcase}}

\n The case conversion procedures are \q{char-downcase} and
\q{char-upcase}:

\q{
(char-downcase #\A) |evalsto #\a
(char-upcase #\a)   |evalsto #\A
}

\index{symbol}

\subsection{Symbols}

\index{self-evaluation}
The simple data types we saw above are {\em
self-evaluating}.  I.e., if you typed any object from these
data types to the listener, the evaluated result returned by
the listener will be the same as what you typed in.

\q{
#t  |evalsto #t
42  |evalsto 42
#\c |evalsto #\c
}

\index{identifier}
\index{variable}

\n Symbols don’t behave the same way.  This is because symbols
are used by Scheme programs as {\em identifiers} for {\em
variables}, and thus will evaluate to the value that the
variable holds.  Nevertheless, symbols are a simple data
type, and symbols are legitimate values that Scheme can
traffic in, along with characters, numbers, and the rest.

\index{quote@\q{quote}}

To specify a symbol without making Scheme think it is a
variable, you should {\em quote} the symbol:


\q{
(quote xyz)
|evalsto xyz
}

\index{'@\q{'} (\q{quote})}

Since this type of quoting is very common in Scheme, a
convenient abbreviation is provided.  The expression

\q{
'E
}

\n will be treated by Scheme as equivalent to

\q{
(quote E)
}

\index{symbol?@\q{symbol?}}

\n Scheme symbols are named by a sequence of characters.  About
the only limitation on a symbol’s name is that it shouldn’t
be mistakable for some other data, e.g., characters or booleans
or numbers or compound data.  Thus, \q{this-is-a-symbol},
\q{i18n},
\q{<=>}, and \q{$!#*} are all symbols; \q{16}, \q{-i} (a
complex number!), %$
\q{#t}, \q{"this-is-a-string"}, and
\q{(barf)} (a list) are not.    The predicate for
checking symbolness is called \q{symbol?}:

\q{
(symbol? 'xyz) |evalsto #t
(symbol? 42)   |evalsto #f
}

\index{symbol!case-insensitivity}

\n Scheme symbols are normally case-insensitive.  Thus the
symbols
\q{Calorie} and \q{calorie} are identical:

\q{
(eqv? 'Calorie 'calorie)
|evalsto #t
}

\index{define@\q{define}}
\index{variable!global}

\n We can use the symbol \q{xyz} as a global variable by using
the form \q{define}:

\q{
(define xyz 9)
}

\n This says the variable \q{xyz} holds the value \q{9}.  If we
feed \q{xyz} to the listener, the result will be the value
held by \q{xyz}:

\q{
xyz
|evalsto 9
}

\index{set"!@\q{set"!}}
\n We can use the form \q{set!} to {\em change} the value held by a
variable:

\q{
(set! xyz #\c)
}

\n Now

\q{
xyz
|evalsto #\c
}

\index{data type!compound}

\section{Compound data types}

Compound data types are built by combining values from
other data types in structured ways.

\index{string}

\subsection{Strings}

Strings are sequences of characters (not to be confused with
symbols, which are simple data that have a sequence of
characters as their name).  You can specify strings by
enclosing the constituent characters in double-quotes.
Strings evaluate to themselves.

\q{
"Hello, World!"
|evalsto "Hello, World!"
}

\index{string@\q{string} (procedure)}

\n The procedure \q{string} takes a bunch of characters and
returns the string made from them:

\q{
(string #\h #\e #\l #\l #\o)
|evalsto "hello"
}

\n Let us now define a global variable \q{greeting}.  

\q{
(define greeting "Hello; Hello!")
}

\n Note that a semicolon inside a string datum does not
trigger a comment.

\index{string-ref@\q{string-ref}}

The characters in a given string can be individually
accessed and modified.  The procedure \q{string-ref} takes a
string and a (0-based) index, and returns the character at
that index:

\q{
(string-ref greeting 0)
|evalsto #\H
}


\index{string-append@\q{string-append}}

\n New strings can be created by appending other strings:

\q{
(string-append "E "
               "Pluribus "
               "Unum")
|evalsto "E Pluribus Unum"
}

\index{make-string@\q{make-string}}

You can make a string of a specified length, and fill it
with the desired characters later.

\q{
(define a-3-char-long-string (make-string 3))
}

\index{string?@\q{string?}}

The predicate for checking stringness is \q{string?}.

\index{string-set"!@\q{string-set"!}}

Strings obtained as a result of calls to \q{string},
\q{make-string}, and \q{string-append} are mutable.
The procedure \q{string-set!} replaces the
character at a given index:

\q{
(define hello (string #\H #\e #\l #\l #\o)) 
hello
|evalsto "Hello"

(string-set! hello 1 #\a)
hello
|evalsto "Hallo"
}
\index{vector}

\subsection{Vectors}

Vectors are sequences like strings, but their elements can
be anything, not just characters.  Indeed, the elements can
be vectors themselves, which is a good way to generate
multidimensional vectors.

\index{vector@\q{vector} (procedure)}

Here’s a way to create a vector of the first five integers:

\q{
(vector 0 1 2 3 4)
|evalsto #(0 1 2 3 4)
}

\n Note Scheme’s representation of a vector value: a \p{#}
character followed by the vector’s contents enclosed in
parentheses.

\index{make-vector@\q{make-vector}}

In analogy with \q{make-string}, the procedure
\q{make-vector} makes a vector of a specific length:

\q{
(define v (make-vector 5))
}

\n The procedures \q{vector-ref} and \q{vector-set!} access and
modify vector elements.
The predicate for checking if something is a vector is \q{vector?}.

\index{dotted pair}
\index{list}
\index{car@\q{car}}
\index{cdr@\q{cdr}}
\index{cons@\q{cons}}

\subsection{Dotted pairs and lists}

A {\em dotted pair} is a compound value made by combining
any two arbitrary values into an ordered couple.  The
first element is called the {\em car}, the second
element is called the {\em cdr}, and the combining
procedure is \q{cons}.

\q{
(cons 1 #t)
|evalsto (1 . #t)
}

\n Dotted pairs are not self-evaluating, and so to specify
them directly as data (i.e., without producing them via
a \q{cons}-call), one must explicitly quote them:

\q{
'(1 . #t) |evalsto (1 . #t)

(1 . #t)  |causeserror
}

\n The accessor procedures are \q{car} and \q{cdr}:

\q{
(define x (cons 1 #t))

(car x)
|evalsto 1

(cdr x)
|evalsto #t
}

\index{set-car"!@\q{set-car"!}}
\index{set-cdr"!@\q{set-cdr"!}}

\n The elements of a dotted pair can be replaced by the
mutator procedures \q{set-car!} and \q{set-cdr!}:

\q{
(set-car! x 2)

(set-cdr! x #f)

x
|evalsto (2 . #f)
}

\n Dotted pairs can contain other dotted pairs.

\q{
(define y (cons (cons 1 2) 3))

y
|evalsto ((1 . 2) . 3)
}

\n The \q{car} of the \q{car} of this list is \q{1}.
The \q{cdr} of the \q{car} of this list is \q{2}.
I.e.,

\q{
(car (car y))
|evalsto 1

(cdr (car y))
|evalsto 2
}

\index{c...r@\q{c...r}}

\n Scheme provides procedure abbreviations for cascaded
compositions of the \q{car} and \q{cdr} procedures.
Thus, \q{caar} stands for “\q{car} of \q{car} of”,
and \q{cdar} stands for “\q{cdr} of \q{car} of”, etc.

\q{
(caar y)
|evalsto 1

(cdar y)
|evalsto 2
}

\n \q{c...r}-style abbreviations for upto four cascades are
guaranteed to exist.  Thus, \q{cadr}, \q{cdadr}, and
\q{cdaddr} are all valid.  \q{cdadadr} might be pushing it.

When nested dotting occurs along the second element,
Scheme uses a special notation to represent the
resulting expression:

\q{
(cons 1 (cons 2 (cons 3 (cons 4 5))))
|evalsto (1 2 3 4 . 5)
}

\n I.e., \q{(1 2 3 4 . 5)} is an abbreviation for \q{(1
. (2 . (3 . (4 . 5))))}.  The last cdr of this
expression is \q{5}.

\index{empty list}

Scheme provides a further abbreviation if the last cdr
is a special object called the {\em empty list}, which
is represented by the expression \q{()}.  The empty
list is not considered self-evaluating, and so one
should quote it when supplying it as a value in a
program:

\q{
'() |evalsto ()
}

\n The abbreviation for a dotted pair of the form \q{(1
. (2 . (3 . (4 . ()))))} is

\q{
(1 2 3 4)
}

\n 
\index{list@\q{list} (procedure)}
This special kind of nested dotted pair is called a
{\em list}.  This particular list is four elements
long.  It could have been created by saying

\q{
(cons 1 (cons 2 (cons 3 (cons 4 '()))))
}

\n but Scheme provides a procedure called \q{list} that
makes list creation more convenient.  \q{list} takes
any number of arguments and returns the list containing
them:

\q{
(list 1 2 3 4)
|evalsto (1 2 3 4)
}

Indeed, if we know all the elements of a list, we can use
\q{quote} to specify the list:

\q{
'(1 2 3 4)
|evalsto (1 2 3 4)
}

\n 
\index{list-ref@\q{list-ref}}
\index{list-tail@\q{list-tail}}
List elements can be accessed by index.

\q{
(define y (list 1 2 3 4))

(list-ref y 0) |evalsto 1
(list-ref y 3) |evalsto 4

(list-tail y 1) |evalsto (2 3 4)
(list-tail y 3) |evalsto (4)
}

\n \q{list-tail} returns the {\em tail} of the list
starting from the given index.

\index{pair?@\q{pair?}}
\index{null?@\q{null?}}
\index{list?@\q{list?}}

The predicates \q{pair?}, \q{list?}, and \q{null?}
check if their argument is a dotted pair, list, or the
empty list, respectively:

\q{
(pair? '(1 . 2)) |evalsto #t
(pair? '(1 2))   |evalsto #t
(pair? '())      |evalsto #f
(list? '())      |evalsto #t
(null? '())      |evalsto #t
(list? '(1 2))   |evalsto #t
(list? '(1 . 2)) |evalsto #f
(null? '(1 2))   |evalsto #f
(null? '(1 . 2)) |evalsto #f
}

\subsection{Conversions between data types}

\index{data type!conversion to and fro}
\index{char->integer@\q{char->integer}}
\index{integer->char@\q{integer->char}}
\index{string->list@\q{string->list}}
\index{list->string@\q{list->string}}
\index{vector->list@\q{vector->list}}
\index{list->vector@\q{list->vector}}
\index{number->string@\q{number->string}}
\index{string->number@\q{string->number}}
Scheme offers many procedures for converting among
the data types.  We already know how to convert between
the character cases using \q{char-downcase} and
\q{char-upcase}.  Characters can be converted into
integers using \q{char->integer}, and integers can be
converted into characters using \q{integer->char}.
(The integer corresponding to a character is usually
its ascii code.)

\q{
(char->integer #\d) |evalsto 100
(integer->char 50)  |evalsto #\2
}

\n Strings can be converted into the corresponding list of
characters.

\q{
(string->list "hello") |evalsto (#\h #\e #\l #\l #\o)
}

\n Other conversion procedures in the same vein are
\q{list->string}, \q{vector->list}, and
\q{list->vector}.

Numbers can be converted to strings:

\q{
(number->string 16) |evalsto "16"
}

Strings can be converted to numbers.  If the string
corresponds to no number, \q{#f} is returned.

\q{
(string->number "16")
|evalsto 16

(string->number "Am I a hot number?")
|evalsto #f
}

\n \q{string->number} takes an optional second argument,
the radix.

\q{
(string->number "16" 8) |evalsto 14
}

\n because \q{16} in base 8 is the number fourteen.

Symbols can be converted to strings, and vice versa:

\q{
(symbol->string 'symbol)
|evalsto "symbol"

(string->symbol "string")
|evalsto string
}

\section{Other data types}

\index{procedure}
Scheme contains some other data types.  One is the
{\em procedure}.  We have already seen many procedures, e.g.,
\q{display}, \q{+}, \q{cons}.  In reality, these are
variables holding the procedure values, which are
themselves not visible as are numbers or characters:

\q{
cons
|evalsto <procedure>
}

\n The procedures we have seen thus far are {\em primitive}
procedures, with standard global variables holding them.
Users can create additional procedure values.

\index{port}

Yet another data type is the {\em port}.  A port is the
conduit through which input and output is performed.
Ports are usually associated with files and consoles.

In our “Hello, World!”\ program, we used the
procedure \q{display} to write a string to the console.
\q{display} can take two arguments, one the value to be
displayed, and the other the output port it should be
displayed on.

In our program, \q{display}’s second argument was
implicit.  The default output port used is the standard
output port.  We can get the current standard output
port via the procedure-call \q{(current-output-port)}.
We could have been more explicit and written

\q{
(display "Hello, World!" (current-output-port))
}

\index{S-expression}

\section{S-expressions}

All the data types discussed here can be lumped
together into a single all-encompassing data type
called the {\em s-expression} ({\em s} for {\em
symbolic}).  Thus \q{42}, \q{#\c}, \q{(1 . 2)}, \q{#(a
b c)}, \q{"Hello"}, \q{(quote xyz)},
\q{(string->number "16")}, and \q{(begin
(display "Hello, World!") (newline))} are all s-expressions.
