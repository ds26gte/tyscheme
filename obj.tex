\chapter{Objects and classes}

\index{object-oriented programming}
\index{object}
\index{class}
\index{instance|see{object}}
\index{method|see{object}}
\index{slot|see{object}}
A {\em class} describes a collection of {\em objects}
that share behavior.  The objects described by a class
are called the {\em instances} of the class.  The class
specifies the names of the {\em slots} that the
instance has, although it is up to the instance to
populate these slots with particular values.
The class also specifies the {\em methods} that can be
applied to its instances.  Slot values can be anything,
but method values must be procedures.  

\index{subclass}
\index{superclass}

Classes are hierarchical.  Thus, a class can be a {\em
subclass} of another class, which is called its {\em
superclass}.  A subclass not only has its own {\em
direct} slots and methods, but also inherits all the
slots and methods of its superclass.  If a class has a
slot or method that has the same name as its
superclass’s, then the subclass’s slot or method is the
one that is retained.

\input obj1.tex

\input obj2.tex

\input mulinh.tex
