\chapter{Nondeterminism}

\index{nondeterminism}
\index{logic programming}
\index{amb@\q{amb}}
McCarthy’s nondeterministic operator
\q{amb} \cite{jmc:amb,wc:amb,zmc:amb} is as old as
Lisp itself, although it is present in no Lisp.
\q{amb} takes zero or more expressions, and makes a
nondeterministic (or “ambiguous”) choice among them,
preferring those choices that cause the program to
converge meaningfully.  Here we will explore an
embedding of \q{amb} in Scheme that makes a depth-first
selection of the ambiguous choices, and uses Scheme’s
control operator \q{call/cc} to backtrack for alternate
choices.  The result is an elegant backtracking
strategy that can be used for searching problem spaces
directly in Scheme without recourse to an extended
language.  The embedding recalls the continuation
strategies used to implement Prolog-style logic
programming \cite{logick,mf:prolog}, but is sparer because the
operator provided is much like a Scheme boolean
operator, does not require special contexts for its
use, and does not rely on linguistic infrastructure
such as logic variables and unification.

\section{Description of \q{amb}}

An accessible description of \q{amb} and many example
uses are found in the premier Scheme textbook
SICP \cite{sicp}.  Informally,
\q{amb} takes zero or more expressions and {\em
nondeterministically} returns the value of {\em one} of
them.  Thus,

\q{
(amb 1 2)
}

\n may evaluate to 1 {\em or} 2.

\q{amb} called with {\em no} expressions has no
value to return, and is considered to {\em fail}.
Thus,

\q{
(amb)
|causeserror amb tree exhausted
}

\n (We will examine the wording of the error message later.)

In particular, \q{amb} is required to return a value if at
least one its subexpressions converges, i.e., doesn’t fail.
Thus,

\q{
(amb 1 (amb))
}

\n and

\q{
(amb (amb) 1)
}

\n both return 1.

Clearly, \q{amb} cannot simply be equated to its first
subexpression, since it has to return a {\em non-failing}
value, if this is at all possible.  However, this is not
all: The bias for convergence is more stringent than a
merely local choice of \q{amb}’s subexpressions.  \q{amb}
should furthermore return {\em that} convergent value
that makes the {\em entire program} converge.  In
denotational parlance, \q{amb} is an {\em angelic}
operator.

For example,

\q{
(amb #f #t)
}

\n may return either \q{#f} or \q{#t}, but in the program

\q{
(if (amb #f #t)
    1
    (amb))
}

\n the first \q{amb}-expression {\em must} return \q{#t}.
If it returned \q{#f}, the \q{if}’s “else” branch would be
chosen, which causes the entire program to fail.

\section{Implementing \q{amb} in Scheme}  

In our implementation of \q{amb}, we will favor
\q{amb}’s subexpressions from left to right.  I.e., the
first subexpression is chosen, and if it leads to overall
failure, the second is picked, and so on.  \q{amb}s occurring
later in the control flow of the program are searched for
alternates before backtracking to previous \q{amb}s.  In
other words, we perform a {\em depth-first} search of the
\q{amb} {\em choice tree}, and whenever we brush against
failure, we backtrack to the most recent node of the tree
that offers a further choice.  (This is called {\em
chronological backtracking.})

We first define a mechanism for setting the base failure
continuation:

\scmfilename amb.scm
\scmdribble{
(define amb-fail '*)

(define initialize-amb-fail
  (lambda ()
    (set! amb-fail
      (lambda ()
        (error "amb tree exhausted")))))

(initialize-amb-fail)
}

\n When \q{amb} fails, it invokes the continuation bound at
the time to \q{amb-fail}.  This is the continuation invoked
when all the alternates in the \q{amb} choice tree have been
tried and were found to fail.

We define \q{amb} as a macro that accepts an indefinite
number of subexpressions.

\scmdribble{
(define-macro amb
  (lambda alts...
    `(let ((+prev-amb-fail amb-fail))
       (call/cc
        (lambda (+sk)

          ,@(map (lambda (alt)
                   `(call/cc
                     (lambda (+fk)
                       (set! amb-fail
                         (lambda ()
                           (set! amb-fail +prev-amb-fail)
                           (+fk 'fail)))
                       (+sk ,alt))))
                 alts...)

          (+prev-amb-fail))))))
}

\n A call to \q{amb} first stores away, in
\q{+prev-amb-fail}, the \q{amb-fail} value that was
current at the time of entry.  This is because the
\q{amb-fail} variable will be set to different failure
continuations as the various alternates are tried.

We then capture the \q{amb}’s {\em entry} continuation \q{+sk}, so
that when one of the alternates evaluates to a non-failing
value, it can immediately exit the \q{amb}.

Each alternate \q{alt} is tried in sequence (the
implicit-\q{begin} sequence of Scheme).

First, we capture the current continuation \q{+fk}, wrap it
in a procedure and set \q{amb-fail} to that procedure.  The
alternate is then evaluated as \q{(+sk alt)}.  If \q{alt}
evaluates without failure, its return value is fed to the
continuation \q{+sk}, which immediately exits the \q{amb}
call.  If \q{alt} fails, it calls \q{amb-fail}.  The first
duty of \q{amb-fail} is to reset \q{amb-fail} to the value
it had at the time of entry.  It then invokes the failure
continuation \q{+fk}, which causes the next alternate, if
any, to be tried.

If all alternates fail, the \q{amb-fail} at \q{amb} entry,
which we had stored in \q{+prev-amb-fail}, is
called.

\section{Using \q{amb} in Scheme} 

To choose a number between 1 and 10, one could say

\q{
(amb 1 2 3 4 5 6 7 8 9 10)
}

\n To be sure, as a program, this will give 1, but
depending on the context, it could return any of the
mentioned numbers.

The procedure \q{number-between} is
a more abstract way to generate numbers from a given \q{lo}
to a given \q{hi} (inclusive):

\q{
(define number-between
  (lambda (lo hi)
    (let loop ((i lo))
      (if (> i hi) (amb)
          (amb i (loop (+ i 1)))))))
}

\n  Thus \q{(number-between 1 6)} will first
generate 1.  Should that fail, the \q{loop} iterates,
producing 2.  Should {\em that} fail, we get 3, and
so on, until 6.  After 6, \q{loop} is called with
the number 7, which being more than 6, invokes
\q{(amb)}, which causes final failure.   (Recall that
\q{(amb)} by itself guarantees
failure.)  At this point, the program containing the
call to
\q{(number-between 1 6)} will backtrack to the
chronologically previous \q{amb}-call, and try to
satisfy {\em that} call in another fashion.

The guaranteed failure of \q{(amb)} can be used to program
{\em assertions}.

\q{
(define assert
  (lambda (pred)
    (if (not pred) (amb))))
}

\n The call \q{(assert pred)} insists that \q{pred} be
true.  Otherwise it will cause the current \q{amb} choice
point to fail.\f{SICP names this procedure
\q{require}.  We use the identifier \q{assert} in order to
avoid confusion with the popular if informal use of
the identifier \q{require} for something else, viz.., an
operator that loads code modules on a per-need basis.}

Here is a procedure using \q{assert} that generates a prime
less than or equal to its argument \q{hi}:

\q{
(define gen-prime
  (lambda (hi)
    (let ((i (number-between 2 hi)))
      (assert (prime? i))
      i)))
}

\n This seems devilishly simple, except that when called as
a program with any number (say 20), it will produce the
uninteresting first solution, i.e., 2.

We would certainly like to get {\em all} the solutions,
not just the first.  In this case, we may want {\em all}
the primes below
20.  One way is to explicitly call the failure
continuation left after the program has produced its first
solution.  Thus,

\q{
(amb)
=> 3
}

\n This leaves yet another failure continuation, which can
be called again for yet another solution:

\q{
(amb)
=> 5
}

\n The problem with this method is that the program is
initially called at the Scheme prompt, and successive
solutions are also obtained by calling \q{amb} at the Scheme
prompt.  In effect, we are using different programs (we
cannot predict how many!), carrying over information from a
previous program to the next.  Instead, we would like to be
able to get these solutions as the return value of a
form that we can call in any context.  To this end, we
define the
\q{bag-of} macro, which returns all
the successful instantiations of its argument.  (If the argument
never succeeds, \q{bag-of} returns the empty list.)
Thus, we could say,

\q{
(bag-of
  (gen-prime 20))
}

\n and it would return

\q{
(2 3 5 7 11 13 17 19)
}

\n The \q{bag-of} macro is defined as follows:

\q{
(define-macro bag-of
  (lambda (e)
    `(let ((+prev-amb-fail amb-fail)
           (+results '()))
       (if (call/cc
            (lambda (+k)
              (set! amb-fail (lambda () (+k #f)))
              (let ((+v ,e))
                (set! +results (cons +v +results))
                (+k #t))))
           (amb-fail))
       (set! amb-fail +prev-amb-fail)
       (reverse! +results))))
}

\n \q{bag-of} first saves away its entry \q{amb-fail}.  It
redefines \q{amb-fail} to a local continuation \q{+k} created within an
\q{if}-test.  Inside the test, the \q{bag-of} argument \q{e}
is evaluated.
If \q{e} succeeds, its result is collected
into a list called \q{+results}, and the local continuation
is called with the value \q{#t}.  This causes the
\q{if}-test to succeed, causing \q{e} to be {\em retried} at its
next backtrack point.  More results for \q{e} are obtained this
way, and they are all collected into \q{+results}.

Finally, when \q{e} fails, it will call the base
\q{amb-fail}, which is simply a call to the local
continuation with the value \q{#f}.  This pushes control
past the \q{if}.   We restore \q{amb-fail} to
its  pre-entry value, and return the \q{+results}.  (The
\q{reverse!} is simply to produce the results in the order
in which they were generated.)

\index{puzzles}
\section{Logic puzzles}

The power of depth-first search coupled
with backtracking becomes obvious when applied to solving
logic puzzles.  These problems are extraordinarily difficult
to solve procedurally, but can be solved concisely and
declaratively with \q{amb}, without taking anything away
from the charm of solving the puzzle.

\subsection{The Kalotan puzzle}

The Kalotans are a tribe with a peculiar quirk.\f{This
puzzle is due to Hunter \cite{hunter}.}  Their males always
tell the truth.  Their females never make two consecutive
true statements, or two consecutive untrue statements.

An anthropologist (let’s call him Worf) has begun to
study them.  Worf does not yet know the Kalotan
language.  One day, he meets a Kalotan (heterosexual)
couple and their child Kibi.  Worf asks Kibi: “Are you
a boy?”  Kibi answers in Kalotan, which of course Worf
doesn’t understand.

Worf turns to the parents (who know English) for
explanation.  One of them says: “Kibi said: ‘I am a
boy.’ ” The other adds: “Kibi is a girl.  Kibi lied.”

Solve for the sex of the parents and Kibi.

\centerline{—}

The solution consists in introducing a bunch of variables,
allowing them to take a choice of values, and
enumerating the conditions on them as a sequence of
\q{assert} expressions.

The variables: \q{parent1},
\q{parent2}, and \q{kibi} are the sexes of the parents (in
order of appearance) and Kibi; \q{kibi-self-desc} is
the sex Kibi claimed to be (in Kalotan); \q{kibi-lied?}
is the boolean on whether Kibi’s claim was a lie.

\q{
(define solve-kalotan-puzzle
  (lambda ()
    (let ((parent1 (amb 'm 'f))
          (parent2 (amb 'm 'f))
          (kibi (amb 'm 'f))
          (kibi-self-desc (amb 'm 'f))
          (kibi-lied? (amb #t #f)))
      (assert
       (distinct? (list parent1 parent2)))
      (assert
       (if (eqv? kibi 'm)
           (not kibi-lied?)))
      (assert
       (if kibi-lied?
           (xor
            (and (eqv? kibi-self-desc 'm)
                 (eqv? kibi 'f))
            (and (eqv? kibi-self-desc 'f)
                 (eqv? kibi 'm)))))
      (assert
       (if (not kibi-lied?)
           (xor
            (and (eqv? kibi-self-desc 'm)
                 (eqv? kibi 'm))
            (and (eqv? kibi-self-desc 'f)
                 (eqv? kibi 'f)))))
      (assert
       (if (eqv? parent1 'm)
           (and
            (eqv? kibi-self-desc 'm)
            (xor
             (and (eqv? kibi 'f)
                  (eqv? kibi-lied? #f))
             (and (eqv? kibi 'm)
                  (eqv? kibi-lied? #t))))))
      (assert
       (if (eqv? parent1 'f)
           (and
            (eqv? kibi 'f)
            (eqv? kibi-lied? #t))))
      (list parent1 parent2 kibi))))
}

\n A note on the helper procedures: The procedure
\q{distinct?} returns true if all the elements in its
argument list are distinct, and false otherwise.  The
procedure \q{xor} returns true if only one of its two
arguments is true, and false otherwise.

Typing \q{(solve-kalotan-puzzle)} will solve the puzzle.

\subsection{Map coloring}

It has been known for some time (but not proven until
1976~\cite{4cp}) that four colors suffice to
color a terrestrial map — i.e., to color the countries
so that neighbors are distinguished.  To actually
assign the colors is still an undertaking, and the
following program shows how nondeterministic
programming can help.

The following program solves the problem of coloring a
map of Western Europe.  The problem and a Prolog
solution are given in {\em The Art of
Prolog} \cite{aop}.  (It is instructive to compare
our solution with the book’s.)

%\input weurope

The procedure \q{choose-color} nondeterministically
returns one of four colors:

\q{
(define choose-color
  (lambda ()
    (amb 'red 'yellow 'blue 'white)))
}

\n In our solution, we create for each country a data
structure.  The data structure is a 3-element list: The
first element of the list is the country’s name; the
second element is its assigned color; and the third
element is the colors of its neighbors.  Note we use
the initial of the country for its color
variable.\f{Spain (Espa\~na) has \q{e} so as not to
clash with Switzerland.}  E.g., the list for Belgium is
\q{(list 'belgium b (list f h l g))}, because — per
the problem statement — the neighbors of Belgium are
France, Holland, Luxembourg, and Germany.

Once we create the lists for each country, we state the
(single!) condition they should satisfy, viz., no
country should have the color of its neighbors.  In
other words, for every country list, the second element
should not be a member of the third element.

\q{
(define color-europe
  (lambda ()

    ;choose colors for each country
    (let ((p (choose-color)) ;Portugal
          (e (choose-color)) ;Spain
          (f (choose-color)) ;France
          (b (choose-color)) ;Belgium
          (h (choose-color)) ;Holland
          (g (choose-color)) ;Germany
          (l (choose-color)) ;Luxemb
          (i (choose-color)) ;Italy
          (s (choose-color)) ;Switz
          (a (choose-color)) ;Austria
          )

      ;construct the adjacency list for
      ;each country: the 1st element is
      ;the name of the country; the 2nd
      ;element is its color; the 3rd
      ;element is the list of its
      ;neighbors’ colors
      (let ((portugal
             (list 'portugal p
                   (list e)))
            (spain
             (list 'spain e
                   (list f p)))
            (france
             (list 'france f
                   (list e i s b g l)))
            (belgium
             (list 'belgium b
                   (list f h l g)))
            (holland
             (list 'holland h
                   (list b g)))
            (germany
             (list 'germany g
                   (list f a s h b l)))
            (luxembourg
             (list 'luxembourg l
                   (list f b g)))
            (italy
             (list 'italy i
                   (list f a s)))
            (switzerland
             (list 'switzerland s
                   (list f i a g)))
            (austria
             (list 'austria a
                   (list i s g))))
        (let ((countries
               (list portugal spain
                     france belgium
                     holland germany
                     luxembourg
                     italy switzerland
                     austria)))

          ;the color of a country
          ;should not be the color of
          ;any of its neighbors
          (for-each
           (lambda (c)
             (assert
              (not (memq (cadr c)
                         (caddr c)))))
           countries)

          ;output the color
          ;assignment
          (for-each
           (lambda (c)
             (display (car c))
             (display " ")
             (display (cadr c))
             (newline))
           countries))))))
}

\n Type \q{(color-europe)} to get a color assignment.
