\scmfilename obj3.scm

\scmwrite{
(load-relative "obj2.scm")
(load-relative "appendmap.scm")
}

\index{inheritance!multiple}
\index{multiple inheritance}

\section{Multiple inheritance}

It is easy to modify the object system to allow classes
to have more than one superclass.  We redefine the
\q{standard-class}  to have a slot called
\q{class-precedence-list} instead of \q{superclass}.
The \q{class-precedence-list} of a class is the list of
{\em all} its superclasses, not just the {\em direct}
superclasses specified during the creation of the class
with \q{create-class}.  The name implies that the
superclasses are listed in a particular order, where
superclasses occurring toward the front of the list
have precedence over the ones in the back of the list.

\scmdribble{
(define standard-class
  (vector 'value-of-standard-class-goes-here
          (list 'slots 'class-precedence-list 'method-names 'method-vector)
          '()
          '(make-instance)
          (vector make-instance)))
}

\scmwrite{
(vector-set! standard-class 0 standard-class)
}

\n Not only has the list of slots changed to include
the new slot, but the erstwhile \q{superclass} slot is
now \q{()} instead of \q{#t}.  This is because the
\q{class-precedence-list} of \q{standard-class} must be
a list.  We could have had its value be \q{(#t)}, but
we will not mention the zero class since it is in every
class’s \q{class-precedence-list}.

The \q{create-class} macro has to modified to accept
a list of direct superclasses instead of a solitary superclass:

\scmdribble{
(define-macro create-class
  (lambda (direct-superclasses slots . methods)
    `(create-class-proc
      (list ,@(map (lambda (su) `,su) direct-superclasses))
      (list ,@(map (lambda (slot) `',slot) slots))
      (list ,@(map (lambda (method) `',(car method)) methods))
      (vector ,@(map (lambda (method) `,(cadr method)) methods))
      )))
}

The \q{create-class-proc} must calculate the class precedence list from
the supplied direct superclasses, and the slot list from the
class precedence list:

\scmdribble{
(define create-class-proc
  (lambda (direct-superclasses slots method-names method-vector)
    (let ((class-precedence-list
           (delete-duplicates
            (append-map
             (lambda (c) (vector-ref c 2))
             direct-superclasses))))
      (send 'make-instance standard-class
            'class-precedence-list class-precedence-list
            'slots
            (delete-duplicates
             (append slots (append-map
                            (lambda (c) (vector-ref c 1))
                            class-precedence-list)))
            'method-names method-names
            'method-vector method-vector))))
}

\n The procedure \q{append-map} is a composition of
  \q{append} and \q{map}:

\q{
(define append-map
  (lambda (f s)
    (let loop ((s s))
      (if (null? s) '()
          (append (f (car s))
                  (loop (cdr s)))))))
}

The procedure \q{send} has to search through the class precedence list
left to right when it hunts for a method.

\scmdribble{
(define send
  (lambda (method-name instance . args)
    (let ((proc
           (let ((class (class-of instance)))
             (if (eqv? class #t) (error 'send)
                 (let loop ((class class)
                            (superclasses (vector-ref class 2)))
                   (let ((k (list-position 
                             method-name
                             (vector-ref class 3))))
                     (cond (k (vector-ref 
                               (vector-ref class 4) k))
                           ((null? superclasses) (error 'send))
                           (else (loop (car superclasses)
                                       (cdr superclasses))))
                     ))))))
      (apply proc instance args))))
}

\scmfilename appendmap.scm

\scmwrite{
(define append-map
  (lambda (f s)
    (let loop ((s s))
      (if (null? s) '()
          (append (f (car s))
                  (loop (cdr s)))))))
}

\scmfilename bike3.scm

\scmwrite{
(define bike-class
  (create-class
   ()
   (frame size parts chain tires)
   (check-fit (lambda (me inseam)
                (let ((bike-size (slot-value me 'size))
                      (ideal-size (* inseam 3/5)))
                  (let ((diff (- bike-size ideal-size)))
                    (cond ((<= -1 diff 1) 'perfect-fit)
                          ((<= -2 diff 2) 'fits-well)
                          ((< diff -2) 'too-small)
                          ((> diff 2) 'too-big))))))))

(define my-bike
  (make-instance bike-class
                 'frame 'titanium
                 'size 21
                 'parts 'ultegra
                 'chain 'sachs
                 'tires 'continental))
}
